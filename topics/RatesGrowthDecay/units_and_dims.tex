% Options for packages loaded elsewhere
\PassOptionsToPackage{unicode}{hyperref}
\PassOptionsToPackage{hyphens}{url}
%
\documentclass[
  ignorenonframetext,
]{beamer}
\usepackage{pgfpages}
\setbeamertemplate{caption}[numbered]
\setbeamertemplate{caption label separator}{: }
\setbeamercolor{caption name}{fg=normal text.fg}
\beamertemplatenavigationsymbolsempty
% Prevent slide breaks in the middle of a paragraph
\widowpenalties 1 10000
\raggedbottom
\setbeamertemplate{part page}{
  \centering
  \begin{beamercolorbox}[sep=16pt,center]{part title}
    \usebeamerfont{part title}\insertpart\par
  \end{beamercolorbox}
}
\setbeamertemplate{section page}{
  \centering
  \begin{beamercolorbox}[sep=12pt,center]{part title}
    \usebeamerfont{section title}\insertsection\par
  \end{beamercolorbox}
}
\setbeamertemplate{subsection page}{
  \centering
  \begin{beamercolorbox}[sep=8pt,center]{part title}
    \usebeamerfont{subsection title}\insertsubsection\par
  \end{beamercolorbox}
}
\AtBeginPart{
  \frame{\partpage}
}
\AtBeginSection{
  \ifbibliography
  \else
    \frame{\sectionpage}
  \fi
}
\AtBeginSubsection{
  \frame{\subsectionpage}
}
\usepackage{amsmath,amssymb}
\usepackage{iftex}
\ifPDFTeX
  \usepackage[T1]{fontenc}
  \usepackage[utf8]{inputenc}
  \usepackage{textcomp} % provide euro and other symbols
\else % if luatex or xetex
  \usepackage{unicode-math} % this also loads fontspec
  \defaultfontfeatures{Scale=MatchLowercase}
  \defaultfontfeatures[\rmfamily]{Ligatures=TeX,Scale=1}
\fi
\usepackage{lmodern}
\ifPDFTeX\else
  % xetex/luatex font selection
\fi
% Use upquote if available, for straight quotes in verbatim environments
\IfFileExists{upquote.sty}{\usepackage{upquote}}{}
\IfFileExists{microtype.sty}{% use microtype if available
  \usepackage[]{microtype}
  \UseMicrotypeSet[protrusion]{basicmath} % disable protrusion for tt fonts
}{}
\makeatletter
\@ifundefined{KOMAClassName}{% if non-KOMA class
  \IfFileExists{parskip.sty}{%
    \usepackage{parskip}
  }{% else
    \setlength{\parindent}{0pt}
    \setlength{\parskip}{6pt plus 2pt minus 1pt}}
}{% if KOMA class
  \KOMAoptions{parskip=half}}
\makeatother
\usepackage{xcolor}
\newif\ifbibliography
\setlength{\emergencystretch}{3em} % prevent overfull lines
\providecommand{\tightlist}{%
  \setlength{\itemsep}{0pt}\setlength{\parskip}{0pt}}
\setcounter{secnumdepth}{-\maxdimen} % remove section numbering
\definecolor{burntorange}{rgb}{0.968, 0.549, 0.114}
\definecolor{burntorangedark}{rgb}{0.486, 0.306, 0.102}
\definecolor{lightblue}{rgb}{0.161, 0.471, 1}
\definecolor{lightbluedark}{rgb}{0.125, 0.271, 0.510}
\definecolor{charcoal}{rgb}{0.21, 0.27, 0.31}
\definecolor{darkgray}{rgb}{0.3, 0.3, 0.3}
\definecolor{darkgrey}{rgb}{0.33, 0.33, 0.33}
\definecolor{cadmiumgreen}{rgb}{0.0, 0.42, 0.24}
\definecolor{brandeisblue}{rgb}{0.0, 0.44, 1.0}
\setbeamercolor{structure}{fg=darkgray}
\setbeamercolor{footline}{fg=darkgray}
\usepackage{amssymb, bm}
\usepackage{amsmath, amsfonts, amscd, epsfig, amssymb, amsthm, adjustbox}
\usepackage{textcomp}
\usepackage{graphicx}
\usepackage{setspace}
\usepackage{enumitem}
\setlist[itemize]{itemsep=9pt, label={--}}
\usepackage{anyfontsize}
\usepackage{ulem}
\usepackage{tcolorbox}[most]
\usepackage{tikz}
\usepackage[T1]{fontenc}
\usepackage{booktabs}
\usepackage{soul}
\setul{4.5pt}{3pt} % First value is thickness, second is distance from text
\usepackage{colortbl}
\usepackage{multirow}
\usepackage{array}
\usepackage{longtable}
\usepackage{listings}
\usepackage{color}
\usepackage{bbold}
\usepackage{mathtools}
\newcolumntype{K}[1]{>{\centering\arraybackslash}p{#1}}
\newcolumntype{Q}[1]{>{\columncolor[gray]{0.8}\centering\arraybackslash}p{#1}}
\newcommand\eho{\stackrel{\mathclap{\small\mbox{$H_0$}}}{=}}
\newcommand\norm[1]{\left\lVert#1\right\rVert}
\newcommand\smalldp{\fontsize{9.4}{7.2}\selectfont}
\newcommand\smalldpp{\fontsize{8.5}{7.2}\selectfont}
\newcommand\smalldppgh{\fontsize{9.5}{7.2}\selectfont}
\newcolumntype{H}{>{\setbox0=\hbox\bgroup}c<{\egroup}@{}}
\newcommand{\bo}[1]{\textcolor{burntorange}{#1}}
\newcommand{\bod}[1]{\textcolor{burntorangedark}{#1}}
\newcommand{\lb}[1]{\textcolor{lightblue}{#1}}
\newcommand{\lbd}[1]{\textcolor{lightbluedark}{#1}}
\newcommand{\dg}[1]{\textcolor{darkgray}{#1}}
\newcommand{\bi}{\begin{itemize}}
\newcommand{\ib}{\end{itemize}}
\newcommand{\p}{\item}
\newcommand{\sk}{\vspace{.5cm}}
\newcommand{\sko}{\vspace{.1in}}
\newcommand{\skoo}{\vspace{.2in}}
\newcommand{\skooo}{\vspace{.3in}}
\newcommand{\skoooo}{\vspace{.05in}}
\newcommand{\hko}{\hspace{.1in}}
\newcommand{\hkoo}{\hspace{.2in}}
\newcommand{\hkooo}{\hspace{.3in}}
\newcommand{\bb}{$\lb{{\small \bullet } }$ \hspace{0.5mm}}
\newcommand{\ba}{$\lb{{\small \rightarrow } }$ \hspace{0.5mm}}
\setbeamertemplate{footline}{\scriptsize{\hfill\insertframenumber\vspace{-.2cm}\hspace*{.35cm}}}
\usepackage{xcolor}
\definecolor{myorange}{RGB}{255, 94, 77}
\setbeamertemplate{footline}{}
\usepackage{ragged2e}
\usepackage{bookmark}
\IfFileExists{xurl.sty}{\usepackage{xurl}}{} % add URL line breaks if available
\urlstyle{same}
\hypersetup{
  pdftitle={Units, Dimension, and Scale},
  pdfauthor={Professors David Ruth and David Puelz},
  hidelinks,
  pdfcreator={LaTeX via pandoc}}

\title{Units, Dimension, and Scale}
\author{Professors David Ruth and David Puelz}
\date{}
\institute{The University of Austin}

\begin{document}
\frame{\titlepage}

\begin{frame}{Units, Dimension, and Scale}
\phantomsection\label{units-dimension-and-scale}
Goals:

\begin{itemize}
\item
  Review notions of unit, dimension, and scale
\item
  Explore arsenic case study
\item
  Program in Python
\end{itemize}
\end{frame}

\begin{frame}{Zaslow Appendix 5 (review/learn)}
\phantomsection\label{zaslow-appendix-5-reviewlearn}
\begin{itemize}
\item
  5.1: Units
\item
  5.2: Scientific notation
\item
  5.3: Scientific notation under operations
\end{itemize}
\end{frame}

\begin{frame}{Some practical instances (estimate value with appropriate
unit)}
\phantomsection\label{some-practical-instances-estimate-value-with-appropriate-unit}
\begin{itemize}
\item
  US Gross Domestic Product
\item
  Avagadro's number
\item
  number of calculations a supercomputer performs in a day
\item
  chance of winning this week's Powerball jackpot
\item
  charge on an electron
\item
  number of cells in a liver
\end{itemize}
\end{frame}

\begin{frame}{Intensive vs.~extensive properties}
\phantomsection\label{intensive-vs.-extensive-properties}
The physical properties of a system can be classified into two
categories:

\begin{itemize}
\item
  \textbf{Extensive properties} depend on the mass of a system.
\item
  \textbf{Intensive properties} are independent of the mass of a system.
\end{itemize}

Examples of each?

\pause

See \url{https://sciencenotes.org/intensive-extensive-properties/} for
more on this.
\end{frame}

\begin{frame}{Zaslow Ch. 4 - Should I worry about arsenic (As) in rice?}
\phantomsection\label{zaslow-ch.-4---should-i-worry-about-arsenic-as-in-rice}
\pause

Do you you have a prior opinion?

\pause

What are the possible responses?

\pause

What would it take for you to favor a particular respone?

\pause

What are the key subquestions to answer?

\pause

\begin{itemize}
\tightlist
\item
  How much As do I ingest associated with rice consumption?
\end{itemize}

\pause

\begin{itemize}
\tightlist
\item
  Is that quantity of As enough to worry about?
\end{itemize}
\end{frame}

\begin{frame}{Example: ppm}
\phantomsection\label{example-ppm}
Quantities of As in water are are measured in \emph{parts per million},
or \emph{ppm}. \pause

\begin{center}
ppm = (mass of solute)/(mass of solution)
\end{center}

\pause

Note: ``ppm'' is technically unitless; i.e.,

\begin{center}
1 ppm = 1 mg/kg
\end{center}

\pause

For water solution with neglible solute mass,

\begin{center}
ppm = 1 mg/L
\end{center}
\end{frame}

\begin{frame}{Zaslow Ch. 4 - Should I worry about arsenic (As) in rice?}
\phantomsection\label{zaslow-ch.-4---should-i-worry-about-arsenic-as-in-rice-1}
\centering

\LARGE  SHIFT TO WHITEBOARD
\end{frame}

\begin{frame}{Python programming}
\phantomsection\label{python-programming}
Two programs today:

\begin{itemize}
\item
  Simple program that triples and input and generates formatted output.
\item
  Documented program that performs conversion similar to assigned
  problem.
\end{itemize}
\end{frame}

\end{document}
