% Options for packages loaded elsewhere
\PassOptionsToPackage{unicode}{hyperref}
\PassOptionsToPackage{hyphens}{url}
%
\documentclass[
  ignorenonframetext,
]{beamer}
\usepackage{pgfpages}
\setbeamertemplate{caption}[numbered]
\setbeamertemplate{caption label separator}{: }
\setbeamercolor{caption name}{fg=normal text.fg}
\beamertemplatenavigationsymbolsempty
% Prevent slide breaks in the middle of a paragraph
\widowpenalties 1 10000
\raggedbottom
\setbeamertemplate{part page}{
  \centering
  \begin{beamercolorbox}[sep=16pt,center]{part title}
    \usebeamerfont{part title}\insertpart\par
  \end{beamercolorbox}
}
\setbeamertemplate{section page}{
  \centering
  \begin{beamercolorbox}[sep=12pt,center]{part title}
    \usebeamerfont{section title}\insertsection\par
  \end{beamercolorbox}
}
\setbeamertemplate{subsection page}{
  \centering
  \begin{beamercolorbox}[sep=8pt,center]{part title}
    \usebeamerfont{subsection title}\insertsubsection\par
  \end{beamercolorbox}
}
\AtBeginPart{
  \frame{\partpage}
}
\AtBeginSection{
  \ifbibliography
  \else
    \frame{\sectionpage}
  \fi
}
\AtBeginSubsection{
  \frame{\subsectionpage}
}
\usepackage{amsmath,amssymb}
\usepackage{iftex}
\ifPDFTeX
  \usepackage[T1]{fontenc}
  \usepackage[utf8]{inputenc}
  \usepackage{textcomp} % provide euro and other symbols
\else % if luatex or xetex
  \usepackage{unicode-math} % this also loads fontspec
  \defaultfontfeatures{Scale=MatchLowercase}
  \defaultfontfeatures[\rmfamily]{Ligatures=TeX,Scale=1}
\fi
\usepackage{lmodern}
\ifPDFTeX\else
  % xetex/luatex font selection
\fi
% Use upquote if available, for straight quotes in verbatim environments
\IfFileExists{upquote.sty}{\usepackage{upquote}}{}
\IfFileExists{microtype.sty}{% use microtype if available
  \usepackage[]{microtype}
  \UseMicrotypeSet[protrusion]{basicmath} % disable protrusion for tt fonts
}{}
\makeatletter
\@ifundefined{KOMAClassName}{% if non-KOMA class
  \IfFileExists{parskip.sty}{%
    \usepackage{parskip}
  }{% else
    \setlength{\parindent}{0pt}
    \setlength{\parskip}{6pt plus 2pt minus 1pt}}
}{% if KOMA class
  \KOMAoptions{parskip=half}}
\makeatother
\usepackage{xcolor}
\newif\ifbibliography
\usepackage{color}
\usepackage{fancyvrb}
\newcommand{\VerbBar}{|}
\newcommand{\VERB}{\Verb[commandchars=\\\{\}]}
\DefineVerbatimEnvironment{Highlighting}{Verbatim}{commandchars=\\\{\}}
% Add ',fontsize=\small' for more characters per line
\usepackage{framed}
\definecolor{shadecolor}{RGB}{248,248,248}
\newenvironment{Shaded}{\begin{snugshade}}{\end{snugshade}}
\newcommand{\AlertTok}[1]{\textcolor[rgb]{0.94,0.16,0.16}{#1}}
\newcommand{\AnnotationTok}[1]{\textcolor[rgb]{0.56,0.35,0.01}{\textbf{\textit{#1}}}}
\newcommand{\AttributeTok}[1]{\textcolor[rgb]{0.13,0.29,0.53}{#1}}
\newcommand{\BaseNTok}[1]{\textcolor[rgb]{0.00,0.00,0.81}{#1}}
\newcommand{\BuiltInTok}[1]{#1}
\newcommand{\CharTok}[1]{\textcolor[rgb]{0.31,0.60,0.02}{#1}}
\newcommand{\CommentTok}[1]{\textcolor[rgb]{0.56,0.35,0.01}{\textit{#1}}}
\newcommand{\CommentVarTok}[1]{\textcolor[rgb]{0.56,0.35,0.01}{\textbf{\textit{#1}}}}
\newcommand{\ConstantTok}[1]{\textcolor[rgb]{0.56,0.35,0.01}{#1}}
\newcommand{\ControlFlowTok}[1]{\textcolor[rgb]{0.13,0.29,0.53}{\textbf{#1}}}
\newcommand{\DataTypeTok}[1]{\textcolor[rgb]{0.13,0.29,0.53}{#1}}
\newcommand{\DecValTok}[1]{\textcolor[rgb]{0.00,0.00,0.81}{#1}}
\newcommand{\DocumentationTok}[1]{\textcolor[rgb]{0.56,0.35,0.01}{\textbf{\textit{#1}}}}
\newcommand{\ErrorTok}[1]{\textcolor[rgb]{0.64,0.00,0.00}{\textbf{#1}}}
\newcommand{\ExtensionTok}[1]{#1}
\newcommand{\FloatTok}[1]{\textcolor[rgb]{0.00,0.00,0.81}{#1}}
\newcommand{\FunctionTok}[1]{\textcolor[rgb]{0.13,0.29,0.53}{\textbf{#1}}}
\newcommand{\ImportTok}[1]{#1}
\newcommand{\InformationTok}[1]{\textcolor[rgb]{0.56,0.35,0.01}{\textbf{\textit{#1}}}}
\newcommand{\KeywordTok}[1]{\textcolor[rgb]{0.13,0.29,0.53}{\textbf{#1}}}
\newcommand{\NormalTok}[1]{#1}
\newcommand{\OperatorTok}[1]{\textcolor[rgb]{0.81,0.36,0.00}{\textbf{#1}}}
\newcommand{\OtherTok}[1]{\textcolor[rgb]{0.56,0.35,0.01}{#1}}
\newcommand{\PreprocessorTok}[1]{\textcolor[rgb]{0.56,0.35,0.01}{\textit{#1}}}
\newcommand{\RegionMarkerTok}[1]{#1}
\newcommand{\SpecialCharTok}[1]{\textcolor[rgb]{0.81,0.36,0.00}{\textbf{#1}}}
\newcommand{\SpecialStringTok}[1]{\textcolor[rgb]{0.31,0.60,0.02}{#1}}
\newcommand{\StringTok}[1]{\textcolor[rgb]{0.31,0.60,0.02}{#1}}
\newcommand{\VariableTok}[1]{\textcolor[rgb]{0.00,0.00,0.00}{#1}}
\newcommand{\VerbatimStringTok}[1]{\textcolor[rgb]{0.31,0.60,0.02}{#1}}
\newcommand{\WarningTok}[1]{\textcolor[rgb]{0.56,0.35,0.01}{\textbf{\textit{#1}}}}
\setlength{\emergencystretch}{3em} % prevent overfull lines
\providecommand{\tightlist}{%
  \setlength{\itemsep}{0pt}\setlength{\parskip}{0pt}}
\setcounter{secnumdepth}{-\maxdimen} % remove section numbering
\definecolor{burntorange}{rgb}{0.968, 0.549, 0.114}
\definecolor{burntorangedark}{rgb}{0.486, 0.306, 0.102}
\definecolor{lightblue}{rgb}{0.161, 0.471, 1}
\definecolor{lightbluedark}{rgb}{0.125, 0.271, 0.510}
\definecolor{charcoal}{rgb}{0.21, 0.27, 0.31}
\definecolor{darkgray}{rgb}{0.3, 0.3, 0.3}
\definecolor{darkgrey}{rgb}{0.33, 0.33, 0.33}
\definecolor{cadmiumgreen}{rgb}{0.0, 0.42, 0.24}
\definecolor{brandeisblue}{rgb}{0.0, 0.44, 1.0}
\setbeamercolor{structure}{fg=darkgray}
\setbeamercolor{footline}{fg=darkgray}
\usepackage{amssymb, bm}
\usepackage{amsmath, amsfonts, amscd, epsfig, amssymb, amsthm, adjustbox}
\usepackage{textcomp}
\usepackage{graphicx}
\usepackage{setspace}
\usepackage{enumitem}
\setlist[itemize]{itemsep=9pt, label={--}}
\usepackage{anyfontsize}
\usepackage{tcolorbox}[most]
\usepackage{tikz}
\usepackage[T1]{fontenc}
\usepackage{booktabs}
\usepackage{colortbl}
\usepackage{multirow}
\usepackage{array}
\usepackage{longtable}
\usepackage{listings}
\usepackage{color}
\usepackage{bbold}
\usepackage{mathtools}
\newcolumntype{K}[1]{>{\centering\arraybackslash}p{#1}}
\newcolumntype{Q}[1]{>{\columncolor[gray]{0.8}\centering\arraybackslash}p{#1}}
\newcommand\eho{\stackrel{\mathclap{\small\mbox{$H_0$}}}{=}}
\newcommand\norm[1]{\left\lVert#1\right\rVert}
\newcommand\smalldp{\fontsize{9.4}{7.2}\selectfont}
\newcommand\smalldpp{\fontsize{8.5}{7.2}\selectfont}
\newcommand\smalldppgh{\fontsize{9.5}{7.2}\selectfont}
\newcolumntype{H}{>{\setbox0=\hbox\bgroup}c<{\egroup}@{}}
\newcommand{\bo}[1]{\textcolor{burntorange}{#1}}
\newcommand{\bod}[1]{\textcolor{burntorangedark}{#1}}
\newcommand{\lb}[1]{\textcolor{lightblue}{#1}}
\newcommand{\lbd}[1]{\textcolor{lightbluedark}{#1}}
\newcommand{\dg}[1]{\textcolor{darkgray}{#1}}
\newcommand{\bi}{\begin{itemize}}
\newcommand{\ib}{\end{itemize}}
\newcommand{\p}{\item}
\newcommand{\sk}{\vspace{.5cm}}
\newcommand{\sko}{\vspace{.1in}}
\newcommand{\skoo}{\vspace{.2in}}
\newcommand{\skooo}{\vspace{.3in}}
\newcommand{\skoooo}{\vspace{.05in}}
\newcommand{\hko}{\hspace{.1in}}
\newcommand{\hkoo}{\hspace{.2in}}
\newcommand{\hkooo}{\hspace{.3in}}
\newcommand{\bb}{$\lb{{\small \bullet } }$ \hspace{0.5mm}}
\newcommand{\ba}{$\lb{{\small \rightarrow } }$ \hspace{0.5mm}}
\setbeamertemplate{footline}{\scriptsize{\hfill\insertframenumber\vspace{-.2cm}\hspace*{.35cm}}}
\usepackage{bookmark}
\IfFileExists{xurl.sty}{\usepackage{xurl}}{} % add URL line breaks if available
\urlstyle{same}
\hypersetup{
  pdftitle={Introduction to Python},
  pdfauthor={(\textbackslash text\{Professor Dave\})\^{}2},
  hidelinks,
  pdfcreator={LaTeX via pandoc}}

\title{Introduction to Python}
\author{\((\text{Professor Dave})^2\)}
\date{}
\institute{The University of Austin}

\begin{document}
\frame{\titlepage}

\begin{frame}{Introduction to Python}
\phantomsection\label{introduction-to-python}
\begin{itemize}
\tightlist
\item
  \textbf{What is Python?}

  \begin{itemize}
  \tightlist
  \item
    Python is a versatile, high-level programming language.
  \item
    Used in web development, data analysis, machine learning,
    automation, and more.
  \end{itemize}

  \vspace{.5cm}
\item
  \textbf{Why Python?}

  \begin{itemize}
  \tightlist
  \item
    Easy to learn and read.
  \item
    Extensive community and resources.
  \item
    Powerful libraries for various fields: science, art, business, and
    more.
  \end{itemize}
\end{itemize}
\end{frame}

\begin{frame}{Value-prop for STEM}
\phantomsection\label{value-prop-for-stem}
\begin{itemize}
\tightlist
\item
  \textbf{Data Science \& Machine Learning}:

  \begin{itemize}
  \tightlist
  \item
    Libraries: NumPy, Pandas, TensorFlow, SciPy.
  \item
    Used for data manipulation, statistical analysis, and building
    models.
  \end{itemize}

  \vspace{.5cm}
\item
  \textbf{Engineering \& Simulations}:

  \begin{itemize}
  \tightlist
  \item
    Used in simulations, optimization, and algorithmic computations.
  \end{itemize}
\end{itemize}
\end{frame}

\begin{frame}{Value-prop for Humanities?}
\phantomsection\label{value-prop-for-humanities}
Yes!

\begin{itemize}
\tightlist
\item
  \textbf{Text Analysis \& Digital Humanities}:

  \begin{itemize}
  \tightlist
  \item
    Libraries: NLTK, spaCy for natural language processing.
  \item
    Analyze large amounts of text for sentiment, themes, and word
    frequencies.
  \end{itemize}

  \vspace{.5cm}
\item
  \textbf{Creative Arts \& Media}:

  \begin{itemize}
  \tightlist
  \item
    Python is used in image processing, media production, and art
    installations.
  \item
    Libraries like PIL (Pillow) and Pygame.
  \end{itemize}
\end{itemize}
\end{frame}

\begin{frame}[fragile]{Basic Python Syntax (Quick Overview)}
\phantomsection\label{basic-python-syntax-quick-overview}
\begin{itemize}
\item
  \textbf{Variables \& Types}:

\begin{Shaded}
\begin{Highlighting}[]
\NormalTok{x }\OperatorTok{=} \DecValTok{10}  \CommentTok{\# Integer}
\NormalTok{name }\OperatorTok{=} \StringTok{"Alice"}  \CommentTok{\# String}
\NormalTok{is\_student }\OperatorTok{=} \VariableTok{True}  \CommentTok{\# Boolean}
\end{Highlighting}
\end{Shaded}
\item
  \textbf{Functions}:

\begin{Shaded}
\begin{Highlighting}[]
\KeywordTok{def}\NormalTok{ greet():}
    \BuiltInTok{print}\NormalTok{(}\StringTok{"Hello, world!"}\NormalTok{)}
\end{Highlighting}
\end{Shaded}
\item
  \textbf{Loops \& Conditionals}:

\begin{Shaded}
\begin{Highlighting}[]
\ControlFlowTok{for}\NormalTok{ i }\KeywordTok{in} \BuiltInTok{range}\NormalTok{(}\DecValTok{5}\NormalTok{):}
    \ControlFlowTok{if}\NormalTok{ i }\OperatorTok{\%} \DecValTok{2} \OperatorTok{==} \DecValTok{0}\NormalTok{:}
        \BuiltInTok{print}\NormalTok{(i, }\StringTok{"is even"}\NormalTok{)}
\end{Highlighting}
\end{Shaded}
\end{itemize}
\end{frame}

\begin{frame}{Installing Python}
\phantomsection\label{installing-python}
\begin{itemize}
\tightlist
\item
  \textbf{Step 1: Download Python}:

  \begin{itemize}
  \tightlist
  \item
    Go to \href{https://python.org}{python.org}.
  \item
    Click ``Downloads'' and select your operating system (Windows,
    macOS, Linux).
  \end{itemize}
\item
  \textbf{Step 2: Install Python}:

  \begin{itemize}
  \tightlist
  \item
    Follow installation instructions on screen.
  \item
    Ensure you check the box to ``Add Python to PATH.''
  \end{itemize}
\end{itemize}
\end{frame}

\begin{frame}[fragile]{Using Python}
\phantomsection\label{using-python}
\begin{itemize}
\tightlist
\item
  \textbf{Step 1: Open a Terminal/Command Prompt}:

  \begin{itemize}
  \tightlist
  \item
    Type \texttt{python} or \texttt{python3} to start the Python
    interpreter.
  \end{itemize}
\item
  \textbf{Step 2: Running Scripts}:

  \begin{itemize}
  \tightlist
  \item
    Create a \texttt{.py} file using a text editor (e.g., Sublime Text,
    Notepad++).
  \item
    Run the script in the terminal: \texttt{python\ script\_name.py}.
  \end{itemize}
\item
  \textbf{Step 3: Using Jupyter Notebooks}:

  \begin{itemize}
  \tightlist
  \item
    Install with \texttt{pip\ install\ notebook}.
  \item
    Start a notebook by typing \texttt{jupyter\ notebook} in your
    terminal.
  \end{itemize}
\end{itemize}
\end{frame}

\end{document}
